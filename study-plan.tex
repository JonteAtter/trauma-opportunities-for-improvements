% Options for packages loaded elsewhere
\PassOptionsToPackage{unicode}{hyperref}
\PassOptionsToPackage{hyphens}{url}
%
\documentclass[
]{article}
\title{Specific opportunities for improvements}
\usepackage{etoolbox}
\makeatletter
\providecommand{\subtitle}[1]{% add subtitle to \maketitle
  \apptocmd{\@title}{\par {\large #1 \par}}{}{}
}
\makeatother
\subtitle{Add a Subtitle if Needed}
\author{Maja Martos}
\date{}

\usepackage{amsmath,amssymb}
\usepackage{lmodern}
\usepackage{iftex}
\ifPDFTeX
  \usepackage[T1]{fontenc}
  \usepackage[utf8]{inputenc}
  \usepackage{textcomp} % provide euro and other symbols
\else % if luatex or xetex
  \usepackage{unicode-math}
  \defaultfontfeatures{Scale=MatchLowercase}
  \defaultfontfeatures[\rmfamily]{Ligatures=TeX,Scale=1}
\fi
% Use upquote if available, for straight quotes in verbatim environments
\IfFileExists{upquote.sty}{\usepackage{upquote}}{}
\IfFileExists{microtype.sty}{% use microtype if available
  \usepackage[]{microtype}
  \UseMicrotypeSet[protrusion]{basicmath} % disable protrusion for tt fonts
}{}
\makeatletter
\@ifundefined{KOMAClassName}{% if non-KOMA class
  \IfFileExists{parskip.sty}{%
    \usepackage{parskip}
  }{% else
    \setlength{\parindent}{0pt}
    \setlength{\parskip}{6pt plus 2pt minus 1pt}}
}{% if KOMA class
  \KOMAoptions{parskip=half}}
\makeatother
\usepackage{xcolor}
\IfFileExists{xurl.sty}{\usepackage{xurl}}{} % add URL line breaks if available
\IfFileExists{bookmark.sty}{\usepackage{bookmark}}{\usepackage{hyperref}}
\hypersetup{
  pdftitle={Specific opportunities for improvements},
  pdfauthor={Maja Martos},
  hidelinks,
  pdfcreator={LaTeX via pandoc}}
\urlstyle{same} % disable monospaced font for URLs
\usepackage[margin=1in]{geometry}
\usepackage{graphicx}
\makeatletter
\def\maxwidth{\ifdim\Gin@nat@width>\linewidth\linewidth\else\Gin@nat@width\fi}
\def\maxheight{\ifdim\Gin@nat@height>\textheight\textheight\else\Gin@nat@height\fi}
\makeatother
% Scale images if necessary, so that they will not overflow the page
% margins by default, and it is still possible to overwrite the defaults
% using explicit options in \includegraphics[width, height, ...]{}
\setkeys{Gin}{width=\maxwidth,height=\maxheight,keepaspectratio}
% Set default figure placement to htbp
\makeatletter
\def\fps@figure{htbp}
\makeatother
\setlength{\emergencystretch}{3em} % prevent overfull lines
\providecommand{\tightlist}{%
  \setlength{\itemsep}{0pt}\setlength{\parskip}{0pt}}
\setcounter{secnumdepth}{-\maxdimen} % remove section numbering
\ifLuaTeX
  \usepackage{selnolig}  % disable illegal ligatures
\fi

\begin{document}
\maketitle

\hypertarget{introduction}{%
\section{Introduction}\label{introduction}}

Trauma, clinically defined as physical injury and the body´s associated
response, cause 4.4 million deaths every year worldwide {[}who2021, ref
funkar inte{]}. However, mortality and morbidity related to trauma have
been significantly reduced in modern countries since the introduction of
trauma systems {[}alharbi2021{]}. Trauma systems have a long tradition
within the military but were not implemented in civil health care until
the 1960s-1970s when the report ``Accidental Death and Disability: The
Neglected Disease of Modern Society'' was published in the US in 1966
{[}choi2021{]}. Since then, trauma systems have been put into practice
in most modern countries with the aim to coordinate and improve
management of critically injured patients, from onset of injury to
high-level care in designated trauma centres {[}nswTraumaSystem{]}.

The American College of Surgeons (ACS) committee on trauma provide
guidelines for the ideal trauma system covering all components of the
system: (I) trauma centres, (II) referral hospitals, (III)
rehabilitation as well as a (IV) data collection and quality
improvement. Quality improvement through continuous evaluation and
identification of opportunities for improvement with subsequent
corrective action plans constitute a cornerstone in the trauma system
and should be systematically proceeded by all trauma centres (5,6).

As injuries differ vastly by feature, risk and what constitute
appropriate care, they should be separated when evaluated for
comparability. For this purpose, the abbreviated injury scale (AIS) has
been implemented as a standardised system to categories all types of
injuries and their severity (7). The AIS is a 6 point scale scoring
system that ranks the severity of injury for five anatomic regions (8).
The ACS has further implemented the AIS system to divide trauma into
different patient cohorts (9).

Opportunities for improvement (OFI), include all aspects of the trauma
system and can be defined as deficiencies at any stage of care that
could be corrected if replaced with more optimized actions. (6) Both
unanticipated (preventable) and anticipated (non-preventable) mortality
can be presented with-or without OFIs. Unanticipated deaths can further
be categorised as preventable or potentially preventable. (7) Two
widespread systems to study unanticipated deaths prevail; autopsy and
multidisciplinary reviews through mortality and morbidity (M\&M)
conferences. (8)

Opportunities for improvement (OFI), include all aspects of the trauma
system and can be defined as deficiencies or aberrations from guidelines
at any stage of care that could be avoided through optimised action (10)
. When trauma leads to death, it can be sorted into either preventable
or non-preventable. In both cases, OFIs can be detected regardless of
whether the outcome could have been prevented. (11) Two widespread
systems to study unanticipated deaths prevail; autopsy and
multidisciplinary reviews through mortality and morbidity (M\&M)
conferences (12).

While autopsy provides information on cause and mechanism of death, it
is costly and not always feasible due to ethical, legal and religious
considerations (13). Neither does it provide information regarding the
process of care. In this respect, M\&M reviews present a more
comprehensive assessment of the course of care (6). However, when
discussing preventability of death, problems arise with respect to the
sensitivity of putting the burden of blame on fellow colleagues and
ambiguity regarding whether mortality is due to inappropriate care or
not. To assess OFIs related to death, therefore constitute a more robust
and feasible method to improve care of trauma patients (12).

To date a variety of studies based on OFIs have been conducted with the
aim to identify recurrent errors for specific patient cohorts or trauma
facilities. Socioeconomic, cultural and geographic issues, trauma
characteristics and healthcare vary between countries and rural/city
areas. In 2020, a study in northern Alberta was conducted with
challenging geography, limited health care resources and a high
proportion of agricultural trauma in mind. Lack of equipment and
personnel were identified as key areas of improvement (14). Other
cohorts focusing on e.g., haemorrhage, which is the leading cause of
trauma related death, have identified specific OFIs for this patient
group (15).

Nordic countries differ from other countries with colder climate, fewer
cases of serious trauma annually and long distances to trauma centres as
few hospitals are equipped to treat trauma-1 patients (16) (17).
Although increasing over the past few years, a systematic review
covering trauma-related studies in Nordic countries the years
1995--2018, show that they fall behind when it comes to number of
publications on the subject compared with other economically similar
countries (16).

In Sweden surgical care is highly centralised and no uniform national
organization for trauma care is at place. This makes evaluation of
competence and performance at site crucial to maintain high quality and
avoid unnecessary risks for the patient (18). Previous studies of
Swedish trauma data have analysed OFIs as clustered variable for
overlapping cohorts. In this study four non-overlapping cohorts were
studied and specific categories of OFIs were found for each cohort.

the patients mean age was 56.48

\hypertarget{methods}{%
\section{Methods}\label{methods}}

\textbf{Study design}

A registry-based cohort study linking data from the Swedish trauma
registry SweTrau and trauma care quality database at the KUH. The
combined data were further assessed through multinominal logistic
regression to identify specific opportunities for improvements (OFI),
identified by the multidisciplinary review board at the KUH All data
were managed and analysed in R software.

\textbf{Aim}

Hitherto, studies of the trauma registry held by the KUH have used the
OFI as a composite measure for all potential lapses leading to
un-optimal care. Although this approach offers insight to whether
opportunities for improvement exist, it is insufficient in providing
health care workers with guidance to actions that may improve care of
trauma patients. Hence, in this study, all specific parameters included
in OFI were analysed individually to analyse their explanatory value of
OFI for four categories of trauma, with the aim to identify the main
areas of improvement.

\textbf{Setting} From 2010, The Swedish Trauma society holds a national
registry over patients suffering serious trauma in Sweden. Patients
included in the registry have either suffered traumatic events leading
to either a trauma alarm or a new injury severity score (NISS) over 15.
In 2021, a total of 10,528 patients were registered (an increase of 17\%
from 2020). Of these, 90\% were assigned to blunt traumas such as falls,
traffic accidents and blunt force traumas with objects and the rest to
penetrating trauma such as gun shots and stabbing. (SweTrau2021)

In Sweden the Karolinska University hospital covers the regions of
Stockholm, Gotland, Södermanland and Västmanland, equivalent to 3
million residents. This is just on pair with the minimum number of
patients needed to be recognized as a quality trauma centre
internationally. The hospital is also the only facility in Sweden to
qualify as a trauma-1 hospital by American standards (19).

To detect non-optimal treatment, treating hospitals evaluate trauma
patients at a M\&M conference held by a multidisciplinary board
appointed by the hospital. The board consists of a surgeon, an
anaesthetist, a trauma nurse and in presence of specific injuries (e.g.,
intracranial, orthopaedical or thoracic/vascular), specialists from
appropriate specialties. Competences involved in the direct care of the
patient are free to attend the conference but should not take part in
the review. (Dödsfallsanalys2021)

Patients are selected to M\&M-conference based on the audit filters
listed below and are thereby selected for review algorithmically and
independent of mortality to study OFIs. If one or more of the audit
filters apply, the patient is manually assessed by a nurse who goes
through the patient journal and makes a final judgement of whether the
patient should be brought to conference. At conference, the
multidisciplinary board determines the cause of death and reviews the
case for suboptimal handling and treatment to identify opportunities for
improvement (20).

Audit filters: +- Systolic blood pressure less than 90 +- Glasgow coma
scale less than 9 and not intubated +- Injury severity score greater
than 15 but not admitted to the intensive care unit +- Time to acute
intervention more than 60 minutes from arrival to hospital +- Time to
computed tomography more than 30 minutes from arrival to hospital +- No
anticoagulant therapy within 72 hours after traumatic brain injury +-
The presence of cardio-pulmonary resuscitation with thoracotomy +- The
presence of a liver or spleen injury +- Massive transfusion, defined as
10 or more units of packed red blood cells within 24 hours.

At the Karolinska University Hospital, results from the conferences are
stored in a local trauma care quality database where all areas of
improvement are registered and collectively stored in a variable, OFI.

\textbf{Study population}

We will study data of patients registered in both the Swedish trauma
registry from SweTrau and the trauma quality data base at the KUH
meeting the following criteria:

+- Older than 15 year +- A NISS \textgreater{} over 15 or a ISS
\textgreater9 +- Being reviewed at an M\&M conference +- Belonging to
one of the following cohorts: + 1. Blunt multisystem trauma with
traumatic brain injury + 2. Blunt multisystem trauma without traumatic
brain injury + 3. Penetrating trauma + 4. Isolated severe traumatic
brain injury

\textbf{Variables}

The primary outcome was opportunities for improvements (OFI) detected by
the M\&M teams at the Karolinska University hospital. The variable is
categorical with the OFIs listed below as possible outcomes. The
exposure were different patient cohorts grouped by mechanism of injury,
(1) blunt multisystem trauma with traumatic brain injury, (2) blunt
multisystem trauma without traumatic brain injury, (3) penetrating
trauma, (4) isolated severe traumatic brain injury. Each cohort studied
was distinguished according to definitions provided by the ACS.

\begin{itemize}
\item
  \begin{enumerate}
  \def\labelenumi{\arabic{enumi}.}
  \tightlist
  \item
    Blunt multisystem trauma: Blunt trauma with injuries of Abbreviated
    Injury Score (AIS) ≥ 3 in at least two of the following AIS body
    regions: head, face, neck, thorax, abdomen, spine, or upper and
    lower extremities.
  \end{enumerate}
\item
  \begin{enumerate}
  \def\labelenumi{\arabic{enumi}.}
  \setcounter{enumi}{1}
  \tightlist
  \item
    Blunt multisystem trauma without traumatic brain injury: Blunt
    trauma with injuries of AIS ≥ 3 limited to only one AIS body region,
    with all other body regions having a maximum AIS ≤
  \end{enumerate}
\item
  \begin{enumerate}
  \def\labelenumi{\arabic{enumi}.}
  \setcounter{enumi}{2}
  \tightlist
  \item
    Penetrating trauma: At least one AIS ≥ 3 injury in any of the
    following AIS body regions: neck, thorax, and abdomen.
  \end{enumerate}
\item
  \begin{enumerate}
  \def\labelenumi{\arabic{enumi}.}
  \setcounter{enumi}{3}
  \tightlist
  \item
    Isolated severe traumatic brain injury: Best GCS within the first 24
    h ≤ 8 OR Best motor score ≤ 3 within the first 24 h, and one of:
  \end{enumerate}
\item
  \begin{enumerate}
  \def\labelenumi{\alph{enumi}.}
  \tightlist
  \item
    Abnormal CT brain (hematoma, contusion, swelling, herniation,
    compression of basal cisterns
  \end{enumerate}
\item
  \begin{enumerate}
  \def\labelenumi{\alph{enumi}.}
  \setcounter{enumi}{1}
  \tightlist
  \item
    Normal CT brain AND (Age \textgreater{} 40 y OR SBP \textless90 mmHg
    on ED arrival) OR posturing (GCS motor = 2, 3)
  \end{enumerate}
\end{itemize}

OFIs Identified at M\&M conference: +- Missed injury +- Problem with
communication +- Inadeqaude competence at site +- Problem at triage +-
Problem with management/ trauma criteria +- No neurosurgeon at site +-
Problem with Tertriry survey after stabilisation/resuscitation\\
+- Problem with management/logistics +- Wrong level of care +-
Inadequate resources +- Problem with logistic and technique +- Exemplary
treatment

The model was adjusted for gender, age and mortalit. All variables with
exception for age were categorical.

\textbf{Data soures/measurement}

The Swedish trauma registry SweTrau includes all trauma patients with a
NISS \textgreater15 or who have triggered an alarm with trauma team
activation in Sweden from 2010 to date. The trauma care quality database
at KUH includes data from trauma patients treated at the hospital from
2014-2021. In the years 2014-2017, patients all random set of patients
with an Injury severity score (ISS) of 9 or higher were included. From
2017, all patients included in the dataset have been reviewed at a M\&M
conference held at the the Karolinska University hospital.

In this study, all patients within the Karolinska University hospital
trauma quality registry reviewed at a M\&M conference were included. For
these patients, data from the Swedish trauma registry by SweTrau were
collected to a merged dataset. The merged dataset was To prevent bias,
the multivariable regression model was developed using a simulated
scrambled dataset with random data. The algorithm for the model was
developed step-by-step and then evaluated by a trained programmer and
statistician before being applied on the real data. Variables such as
ID-number and name were scrambled and anonymised throughout analysis of
the real dataset as well. then divided into four cohorts; (1) blunt
multisystem trauma with traumatic brain injury, (2) blunt multisystem
trauma without traumatic brain injury, (3) penetrating trauma, (4)
isolated severe traumatic brain injury

\textbf{Bias}

To prevent bias, the multivariable regression model will be developed
using a simulated scrambled dataset with random data. The algorithm for
the model will be developed step-by-step and then evaluated by a trained
programmer and statistician before being applied on theTo prevent bias,
the multivariable regression model was developed using a simulated
scrambled dataset with random data. The algorithm for the model was
developed step-by-step and then evaluated by a trained programmer and
statistician before being applied on the real data. Variables such as
ID-number and name were scrambled and anonymised throughout analysis of
the real dataset as well. real data. Variables such as ID-number and
name will be scrambled and anonymised throughout analysis of the real
dataset as well.

\textbf{Study size}

\end{document}
